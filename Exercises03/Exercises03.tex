%----------------------------------------------------------------------------------------
%	PACKAGES AND OTHER DOCUMENT CONFIGURATIONS
%----------------------------------------------------------------------------------------

\documentclass{article}

\usepackage{fancyhdr} % Required for custom headers
\usepackage{lastpage} % Required to determine the last page for the footer
\usepackage{extramarks} % Required for headers and footers
\usepackage[usenames,dvipsnames]{color} % Required for custom colors
\usepackage{graphicx} % Required to insert images
\usepackage{amsmath}% http://ctan.org/pkg/amsmath
\usepackage{listings} % Required for insertion of code
%\usepackage{couriernew} % Required for the courier font

\usepackage{enumerate} % Required for enumerating with letters

\usepackage{mathpazo}
\usepackage{avant}
\usepackage{inconsolata}

\newcommand{\pN}{\mathcal{N}}
\newcommand{\iid}{\overset{\text{iid}}{\sim}}
\newcommand\independent{\protect\mathpalette{\protect\independenT}{\perp}}
\def\independenT#1#2{\mathrel{\rlap{$#1#2$}\mkern2mu{#1#2}}}

% Margins
\topmargin=-0.45in
\evensidemargin=0in
\oddsidemargin=0in
\textwidth=6.5in
\textheight=9.0in
\headsep=0.25in

\linespread{1.1} % Line spacing

% Set up the header and footer
\pagestyle{fancy}
\lhead{\hmwkAuthorName} % Top left header
\chead{\hmwkClass\ : \hmwkTitle} % Top center head
\rhead{} % Top right header
\lfoot{\lastxmark} % Bottom left footer
\cfoot{} % Bottom center footer
\rfoot{Page\ \thepage\ of\ \protect\pageref{LastPage}} % Bottom right footer
\renewcommand\headrulewidth{0.4pt} % Size of the header rule
\renewcommand\footrulewidth{0.4pt} % Size of the footer rule

\setlength\parindent{0pt} % Removes all indentation from paragraphs

%----------------------------------------------------------------------------------------
%	CODE INCLUSION CONFIGURATION
%----------------------------------------------------------------------------------------

\definecolor{MyDarkGreen}{rgb}{0.0,0.4,0.0} % This is the color used for comments
\lstloadlanguages{R} % Load R syntax for listings, for a list of other languages supported see: ftp://ftp.tex.ac.uk/tex-archive/macros/latex/contrib/listings/listings.pdf
\lstset{language=R, % Use R in this example
        frame=single, % Single frame around code
        basicstyle=\small\ttfamily, % Use small true type font
        keywordstyle=[1]\color{Blue}, % Perl functions bold and blue
        keywordstyle=[2]\color{Purple}, % Perl function arguments purple
        keywordstyle=[3]\color{Blue}\underbar, % Custom functions underlined and blue
        identifierstyle=, % Nothing special about identifiers                                         
        commentstyle=\usefont{T1}{pcr}{m}{sl}\color{MyDarkGreen}\small, % Comments small dark green courier font
        stringstyle=\color{Purple}, % Strings are purple
        showstringspaces=false, % Don't put marks in string spaces
        tabsize=4, % 5 spaces per tab
        %
        % Put standard Perl functions not included in the default language here
        morekeywords={rand},
        %
        % Put Perl function parameters here
        morekeywords=[2]{on, off, interp},
        %
        % Put user defined functions here
        morekeywords=[3]{test},
       	%
        morecomment=[l][\color{Blue}]{...}, % Line continuation (...) like blue comment
        numbers=left, % Line numbers on left
        firstnumber=1, % Line numbers start with line 1
        numberstyle=\tiny\color{Blue}, % Line numbers are blue and small
        stepnumber=5 % Line numbers go in steps of 5
}

% Creates a new command to include a perl script, the first parameter is the filename of the script (without .pl), the second parameter is the caption
\newcommand{\rscript}[2]{
\begin{itemize}
\item[]\lstinputlisting[caption=#2,label=#1]{#1.r}
\end{itemize}
}

%----------------------------------------------------------------------------------------
%	DOCUMENT STRUCTURE COMMANDS
%	Skip this unless you know what you're doing
%----------------------------------------------------------------------------------------

% Header and footer for when a page split occurs within a problem environment
\newcommand{\enterProblemHeader}[1]{
\nobreak\extramarks{#1}{#1 continued on next page\ldots}\nobreak
\nobreak\extramarks{#1 (continued)}{#1 continued on next page\ldots}\nobreak
}

% Header and footer for when a page split occurs between problem environments
\newcommand{\exitProblemHeader}[1]{
\nobreak\extramarks{#1 (continued)}{#1 continued on next page\ldots}\nobreak
\nobreak\extramarks{#1}{}\nobreak
}

\setcounter{secnumdepth}{0} % Removes default section numbers
\newcounter{homeworkProblemCounter} % Creates a counter to keep track of the number of problems

\newcommand{\homeworkProblemName}{}
\newenvironment{homeworkProblem}[1][Problem \arabic{homeworkProblemCounter}]{ % Makes a new environment called homeworkProblem which takes 1 argument (custom name) but the default is "Problem #"
\stepcounter{homeworkProblemCounter} % Increase counter for number of problems
\renewcommand{\homeworkProblemName}{#1} % Assign \homeworkProblemName the name of the problem
\section{\homeworkProblemName} % Make a section in the document with the custom problem count
\enterProblemHeader{\homeworkProblemName} % Header and footer within the environment
}{
\exitProblemHeader{\homeworkProblemName} % Header and footer after the environment
}

\newcommand{\problemAnswer}[1]{ % Defines the problem answer command with the content as the only argument
\noindent\framebox[\columnwidth][c]{\begin{minipage}{0.98\columnwidth}#1\end{minipage}} % Makes the box around the problem answer and puts the content inside
}

\newcommand{\homeworkSectionName}{}
\newenvironment{homeworkSection}[1]{ % New environment for sections within homework problems, takes 1 argument - the name of the section
\renewcommand{\homeworkSectionName}{#1} % Assign \homeworkSectionName to the name of the section from the environment argument
\subsection{\homeworkSectionName} % Make a subsection with the custom name of the subsection
\enterProblemHeader{\homeworkProblemName\ [\homeworkSectionName]} % Header and footer within the environment
}{
\enterProblemHeader{\homeworkProblemName} % Header and footer after the environment
}

%----------------------------------------------------------------------------------------
%	NAME AND CLASS SECTION
%----------------------------------------------------------------------------------------

\newcommand{\hmwkTitle}{Exercises 3 -- Linear smoothing and Gaussian processes} % Assignment title
\newcommand{\hmwkDueDate}{\today} % Due date
\newcommand{\hmwkClass}{SDS\ 383D} % Course/class
\newcommand{\hmwkClassTime}{} % Class/lecture time
\newcommand{\hmwkClassInstructor}{Professor Scott} % Teacher/lecturer
\newcommand{\hmwkAuthorName}{Spencer Woody} % Your name

%----------------------------------------------------------------------------------------
%	TITLE PAGE
%----------------------------------------------------------------------------------------

\title{
\vspace{2in}
\textmd{\textbf{\hmwkClass:\ \hmwkTitle}}\\
\normalsize\vspace{0.1in}\small{\hmwkDueDate}\\
\vspace{0.1in}\large{\textit{\hmwkClassInstructor\ }}
\vspace{3in}
}

\author{\textbf{\hmwkAuthorName}}
\date{} % Insert date here if you want it to appear below your name

%----------------------------------------------------------------------------------------

\begin{document}

\maketitle

\newpage

%----------------------------------------------------------------------------------------
%	PROBLEM 1
%----------------------------------------------------------------------------------------

% To have just one problem per page, simply put a \clearpage after each problem

\begin{homeworkProblem}

\large
\textbf{Basic Concepts}
\normalsize


\begin{enumerate}[(A)]
	\item % A
	%
	%
	%
	%
	%
	%
	%
\end{enumerate}




\end{homeworkProblem}

%----------------------------------------------------------------------------------------
%	PROBLEM 2
%----------------------------------------------------------------------------------------

% To have just one problem per page, simply put a \clearpage after each problem

\pagebreak

\begin{homeworkProblem}

\large
\textbf{Curve fitting by linear smoothing}
\normalsize

In this problem, consider a general nonlinear regression with one predictor and one response, $y_i = f(x_i) + \epsilon_i$, where $\epsilon_i$ are mean-zero random variables.

\begin{enumerate}[(A)]
	\item % A
	%
	%
	%
	For now, consider a linear regression on a response $y_i$ with one predictor $x_i$, and both $y_i$ and $x_i$ have had their means subtracted, so the $y_i = \beta x_i + \epsilon_i$. Define $S_x := \sum_{i=1}^n x_i^2.$ The least squares estimate for the coefficient, from Exercises 1, is 
	%
	%
	%
	\begin{align*}
		\hat{\beta} &= (X^TX)^{-1}X^Ty \\
			 		&= (x^Tx)^{-1}x^T y \\
					&= \frac{\sum_{i=1}^n x_i \cdot y_i}{\sum_{i=1}^n x_i^2} \\
					&= \frac{\sum_{i=1}^n x_i \cdot y_i}{S_x} \\ 
					&= \sum_{i=1}^n \frac{x_i}{S_x} \cdot y_i.
	\end{align*}
	%
	%
	%
	So now our prediction $y^\star | x^\star$ is, 
	%
	%
	%
	\begin{align*}
		\hat{y}^\star &= \hat{f}(x^\star) \\
		&= \hat{\beta}x^\star \\
		&= \left( \sum_{i=1}^n \frac{x_i}{S_x} \cdot y_i \right) \cdot x^\star \\
		&= \sum_{i=1}^n \left( \frac{x_i}{S_x} \cdot x^\star \right) \cdot y_i,
	\end{align*}
	%
	%
	%
	which we recognize as being in the form of the general \emph{linear smoother}
	%
	%
	%
	\begin{align*}
		\hat{f}\left( x^\star \right) &= \sum_{i=1}^n w(x_i, x^\star) \cdot y_i
	\end{align*}
	%
	%
	%
	for some weight function $w(x_i, x^\star)$. In particular, the weight function for linear regression gives weight to each $y_i$ proportional to the value of $x_i$. Contrast this with the $k$-nearest neighbors smoothing weight function,
	%
	%
	%
	\begin{align*}
		w_K(x_i, x^\star) &= \begin{cases}
			1/K & \text{if } x_i \text{ is one of the K closest sample points to } x^\star \\
			0 & \text{otherwise} 
		\end{cases},
	\end{align*}
	%
	%
	%
	which gives \emph{equal} weight to $y_i$s but \emph{only} to the $k$-nearest neighbors of $x^\star$.
	%
	%
	%
	\item % B
	%
	%
	%
	Now we have the very general weight function
	%
	%
	%
	\begin{align*}
		w(x_i, x^\star) &= \frac{1}{h} \cdot K\left( \frac{x_i - x^\star}{h}, \right)
	\end{align*}
	%
	%
	%
	where $K(\bullet)$ is some kernel function. The script \texttt{myfuns03.R} in the appendix shows an \textsf{R} function for linear smoothing, as well functions for the uniform and Gaussian kernels.
	%
	%
	%
\end{enumerate}

\end{homeworkProblem}

%----------------------------------------------------------------------------------------
%	PROBLEM 3
%----------------------------------------------------------------------------------------

\pagebreak

% To have just one problem per page, simply put a \clearpage after each problem

\begin{homeworkProblem}

\large
\textbf{Cross validation}
\normalsize


\begin{enumerate}[(A)]
	\item % A
	%
	%
	%
	%
	%
	%
	%
\end{enumerate}

\end{homeworkProblem}

%----------------------------------------------------------------------------------------
%	PROBLEM 4
%----------------------------------------------------------------------------------------

\pagebreak

% To have just one problem per page, simply put a \clearpage after each problem

\begin{homeworkProblem}

\large
\textbf{Local polynomial regression}
\normalsize


\begin{enumerate}[(A)]
	\item % A
	%
	%
	%
	%
	%
	%
	%
\end{enumerate}

\end{homeworkProblem}

%----------------------------------------------------------------------------------------
%	PROBLEM 5
%----------------------------------------------------------------------------------------

\pagebreak

% To have just one problem per page, simply put a \clearpage after each problem

\begin{homeworkProblem}

\large
\textbf{Gaussian processes}
\normalsize


\begin{enumerate}[(A)]
	\item % A
	%
	%
	%
	%
	%
	%
	%
\end{enumerate}

\end{homeworkProblem}


%----------------------------------------------------------------------------------------
%	PROBLEM 6
%----------------------------------------------------------------------------------------

\pagebreak

% To have just one problem per page, simply put a \clearpage after each problem

\begin{homeworkProblem}

\large
\textbf{In nonparametric regression and spacial smoothing}
\normalsize


\begin{enumerate}[(A)]
	\item % A
	%
	%
	%
	%
	%
	%
	%
\end{enumerate}

\end{homeworkProblem}


%%----------------------------------------------------------------------------------------
%%	LIST CODE
%%----------------------------------------------------------------------------------------

\pagebreak
% \rscript{homework03.r}{Sample Perl Script With Highlighting}
R code for \texttt{myfuns03.R}
\lstinputlisting[language=R]{myfuns03.R}
\pagebreak
R code for \texttt{exercises03.R}
\lstinputlisting[language=R]{exercises03.R}


%----------------------------------------------------------------------------------------

\end{document}