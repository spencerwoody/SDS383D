\title{Peer Review 2 for Jennifer}
\author{
        Spencer Woody \\
                SDS 383D
}
\date{April 12, 2017}

\documentclass[11pt]{article}

\usepackage[margin=1.6in]{geometry}
\usepackage{graphicx}
\usepackage{amsmath}
\usepackage{listings}
\usepackage{hyperref}
\hypersetup{
    colorlinks = true%,
%    linkbordercolor = {white}
}
\usepackage[]{algorithm2e}
\usepackage{mathpazo}
\usepackage{inconsolata}

\begin{document}

\maketitle

\setlength{\parindent}{0cm}

\section{Introduction}



Hi Jennifer, thanks for letting me review your write-up and code. This write-up will just list the things we discussed the other day, and give links to specific resources I mentioned.

\section{Tips for \textsf{R}}

\begin{itemize}
	\item 
	In your Gibbs sampler, I noticed that you are sampling the covariance matrix $\Sigma$ from the inverse-Wishart distribution using the \texttt{MCMCpack} package at each step and then inverting it to sample from the full conditional of each $b_i$. Instead of doing this, you can sample the precision matrix $\Sigma^{-1}$ from the Wishart using \href{https://stat.ethz.ch/R-manual/R-devel/library/stats/html/rWishart.html}{\texttt{rWishart}} command, which is also in base \textsf{R} so you do not have to load any special package for it.
	\item
	Also relating to your Gibbs sampler, I suggest you initialize your vectors, matrices, and arrays for storing MCMC samples from the posterior with \texttt{NA} values instead of each element being 0. For example, the command \begin{verbatim}
			matrix(nrow = n, ncol = p)
	\end{verbatim} will give a $n \times p$ matrix of all \texttt{NA} values. This will help save memory and also make it easier to find bugs in your code.
	\item
	Since you are looping over the $b_i$'s, I suggest you try using the \href{https://cran.r-project.org/web/packages/foreach/vignettes/foreach.pdf}{\texttt{foreach} package}), which allows you to parallelize loop tasks. Since you have a MacBook Pro, this should cut your computation time by a significant amount. The link I gave goes in depth on how to use the package.
	\item
	Printing the iteration to your console each time can slow down your Gibbs sampler a bit and also crowd out other output you may want to go back to, so instead you can add an \texttt{if} statement and the modulo command (e.g. \texttt{a \%\% n} is the same as $a \text{ mod } n$) to print, say, only each 1000th iteration. Speaking of which,\ldots
	\item
	Use \href{http://www.cookbook-r.com/Strings/Creating_strings_from_variables/}{\texttt{sprintf}} in \textsf{R} to create strings from variables. It's very handy and you can also use this in \textsf{C}! 
	\item 
	This is a fastidious bit of feedback, but I strongly suggest following certain guidelines for your code, like always having a space after commas. The \href{https://google.github.io/styleguide/Rguide.xml}{Google \textsf{R} Style Guide} is my Bible.
\end{itemize}

\section{Tips for the write-up}

\subsection{Math notation}

\begin{itemize}
	\item Use the \texttt{\textbackslash times} command in \LaTeX to make the $\times$ symbol. 
	\item Use the \texttt{\textbackslash log} command in math environments to produce an upright $\log$ sign (in fact, you can find all such commands listed \href{http://web.ift.uib.no/Teori/KURS/WRK/TeX/symALL.html}{here} under Table 8 [I didn't even know that \texttt{\textbackslash Pr} existed but I'll be using that now]).
\end{itemize}

\subsection{Visualization}

\begin{itemize}
	\item Be sure to have axis labels for all your graphs, and if you have them side-by-side, it is also a good idea to have them on the same scale (which is admittedly hard to do on the linear scale using this dataset).
	\item An informative plot for model checking in this plotting the posterior predictive of log-volume, e.g. plotting the posterior predictive median at each log-price as a line and then have a \href{http://ggplot2.tidyverse.org/reference/geom_ribbon.html}{\texttt{geom\_ribbon}} for a 95\% predictive interval.
	\item We've talked about this at length but if you still want help using \texttt{ggplot2} for facet plots just let me know. $\ddot\smile$
\end{itemize}

\section{Conclusion}

Overall, great job with this assignment. I hope this review has been helpful for you, and do not hesitate to ask me to clarify anything or give any other tips. \\

Spencer

\end{document}

